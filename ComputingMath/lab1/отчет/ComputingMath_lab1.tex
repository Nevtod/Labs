\documentclass{llncs}
\usepackage{cmap}
\usepackage[T2A,T1]{fontenc}
\usepackage[utf8]{inputenc}
\usepackage[russian]{babel}
\usepackage[pdftex]{graphicx}
\usepackage{comment}
\usepackage{cite}
\usepackage{setspace}
\usepackage{authblk}
\usepackage{calc}

\usepackage{amsmath}
\usepackage{tikz}
\usepackage{fancyref}
\usepackage{amssymb}
\usepackage{verbatim}
\usepackage{pdfpages}
\usepackage{subcaption}
\renewcommand\thesubfigure{\asbuk{subfigure}}
\captionsetup{compatibility=false}
\captionsetup[figure]{labelfont=bf}
\captionsetup[subfigure]{labelfont=normalfont}
\usetikzlibrary{arrows}
\usetikzlibrary{patterns}

\usepackage{graphicx}
\graphicspath{{images/},{plots/}}
\newcommand*{\hm}[1]{#1\nobreak\discretionary{}
{\hbox{$\mathsurround=0pt #1$}}{}}

\title{Отчёт по лаб. работе № 1}
\author{
Парошин Владислав
}
\begin{document}
\maketitle

\section{ Задача 1}

	Выпишем систему уравнений

\begin{align*}
-u_7 -u_{11} + 4*u_6 = h^2 \\
-u_6 -u_{12} -u_{16} + 4*u_{11} = h^2 \\
-u_{11} -u_{17} + 4*u_{16} = h^2 \\
-u_6 -u_8 -u_{12} + 4*u_7 = h^2 \\
-u_{11} -u_7 -u_{13} -u_{17} + 4*u_{12} = h^2 \\
-u_{16} -u_{12} -u_{18} + 4*u_{17} = h^2 \\
-u_7 -u_{13} + 4*u_8 = h^2 \\
-u_{12} -u_8 -u_{18} + 4*u_{13} = h^2 \\
-u_{17} -u_{13} + 4*u_{18} = h^2
\end{align*}

В матричном виде:

\begin{figure}[]
	\centering
	\includegraphics[scale=0.3]{matrix_text.png}
	%\caption{\label{fig:diagram} }
\end{figure}


Визуализация:


\begin{figure}[]
	\centering
	\includegraphics[width=\textwidth]{matrix.png}
	%\caption{\label{fig:diagram} }
\end{figure}

и решение системы

\[
  u = 
  \begin{pmatrix}
    0.04296875 \\
    0.0546875  \\
    0.04296875 \\
    0.0546875  \\
    0.0703125  \\
    0.0546875  \\
    0.04296875 \\
    0.0546875  \\
    0.04296875
  \end{pmatrix}
\]

\section{ Задача 2}
	
	\begin{figure}[]
	\centering
	\includegraphics[width=\textwidth]{matrix_big.png}
	\caption{ Визуализируем построенную матрицу}
\end{figure}
	
	
	\begin{figure}[]
	\centering
	\includegraphics[width=\textwidth]{solution.png}
	\caption{Визуализируем решение}
\end{figure}

\newpage
Теперь посмотрим на сходимость разных методов
\begin{figure}[]
	\centering
	\includegraphics[width=\textwidth]{zeidel.png}
	\caption{Метод Зейделя сходится быстро}
\end{figure}


\begin{figure}[]
	\centering
	\includegraphics[width=\textwidth]{jacobi.png}
	\caption{Метод Якоби расходится быстро}
\end{figure}


\begin{figure}[]
	\centering
	\includegraphics[width=\textwidth]{mpi.png}
	\caption{
Метод ПИ Сходится достаточно медленно, при условии, что выбран оптимальный параметр $tau = \frac{2}{\lambda_max + \lambda_min}$}
\end{figure}


\end{document}
